  
 
 \phantom \quad \\
  \newpage
   \thispagestyle{empty}
   \lhead{Teaching Philosophy}
\hrule \phantom \quad  \vspace*{1\baselineskip}  \\
{\bf Teaching Philosophy}
 \vspace*{1\baselineskip}  \hrule \phantom \quad \\
\vspace*{3\baselineskip} \phantom \quad \\
{\bf Inspiration:}
Growing up in the presence of my father, a dedicated teacher, profoundly influenced my educational philosophy. He was a masterful guide who fueled my passion for learning by posing thought-provoking questions, stimulating my curiosity and shaping my approach to critical thinking. Instead of providing direct answers, he encouraged me to explore and discover solutions independently. 

This formative experience became the bedrock of my teaching philosophy, emphasizing the important role of critical thinking in education. My aim is to gradually but firmly establish curiosity and foster creativity, ultimately transforming students into innovative and knowledgeable individuals.
 \\ [0.2cm]
{\bf Inquiry: }
I believe the most effective way to engage students is through inquiry-based learning. Students are more likely to embrace a new concept when they perceive it as a solution to a problem in their minds. The approach involves initiating the class with well-chosen questions and allowing students to explore the topic with guidance from the teacher. 

In this method, I express the fundamental problem, instill a desire to find the answer, and facilitate class discussions. My role is that of a moderator leading students through a challenging experience designed to promote discussion, critical thinking, and discovery. For instance, when starting Digital Logic Design at the beginning of the semester and discussing number systems, I began with the question, "Why is the decimal system predominant, and not, for instance, octal or hexadecimal?" 

After discussing this, I posed the next question, "Now, why do machines use the binary number system?" I noticed more students engaged in answering the second question when the first question was discussed beforehand, compared to classes where I directly presented the second question. This method also fosters a conceptual understanding of problems and solutions in electrical engineering, enabling students to connect different concepts and apply engineering methods creatively.
    \\[0.2cm]
   {\bf A Deep Understanding:}
   A successful teaching method involves a combination of various techniques. Another technique I find particularly useful is cognitive learning, where the focus is on understanding the subject at a deeper level rather than memorizing approaches to specific problems. My approach to this process includes three elements: first, students need to understand why they are learning the subject;
    second, they should gain knowledge about the subject at a deep level; and third, they should contemplate the application of what they learned. 
    
    Course contents should be designed to ensure students acquire a deep understanding and develop the skills to apply their knowledge in their field. For example, when presenting logical operations (AND, OR, and NOT) in a Digital Logic Design class, I explained the concept of an idealized switch network, gave simple examples, and asked students to identify different combinations and create separate components. Subsequently, I introduced Boolean algebra and logical operations as tools to analyze and design circuits algebraically in terms of logic gates.
    \\[0.2cm]
   {\bf Creativity:}
Arguably the most crucial aspect of learning, especially at the highest level, is creativity. The scientific approach to creativity involves studying and mastering existing achievements, asking questions, proposing solutions, designing experiments to validate proposed solutions, and communicating research with other scientists to receive feedback. 

My plan to involve graduate and last-year undergraduate students in research includes providing them with sufficient background and knowledge in a specific area I believe has research potential. I will ask them to identify and compile a list of new and noteworthy works published in the subject. Subsequently, they will write briefs on those papers, highlighting their contributions.

 The next step involves presenting the papers to deepen their understanding of the works. I firmly believe that explaining something to someone else enhances insight into the topic. Therefore, such conferences will be a fundamental part of my approach, helping students evaluate literature and communicate their findings. The subsequent phase will include a critical analysis of the papers, requiring students to consider the weaknesses, limitations, and drawbacks of proposed approaches. Conducting simulations to replicate and test findings from the papers will help them learn necessary tools and approaches, facilitating problem-spotting.

At  each phase, it is my responsibility as a teacher to evaluate their work, provide feedback, and offer suggestions for improvement. This process allows students to enhance their ability to identify problems, review literature for solutions, propose new solutions, and effectively communicate their research.

In summary, I aspire to impart the qualities that meant the most to me during my education: inquiry, deep understanding, and creativity. I aim to instill a love for engineering in my students, as I personally feel an infinite passion for it
