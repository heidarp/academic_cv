\newpage
  \thispagestyle{empty}
 \lhead{Record of Teaching Effectiveness}
 \phantom \quad \\
\hrule \phantom \quad  \vspace*{1\baselineskip}  \\
 {\bf Record of Teaching Effectiveness}
 \vspace*{1\baselineskip}  \hrule \phantom \quad \\
\vspace*{3\baselineskip} \phantom \quad \\
{\bf Teaching at High School:}\\ 

My first teaching experience was at a high school. I underwent a four-month intensive training before I started teaching, where I took courses on teaching methods, student evaluation, psychology, and the rules and regulations of teaching. I passed these courses with high grades and was very enthusiastic to start teaching and I was hoping that I would be a very good teacher.  I prepared slides and course materials that went beyond the textbook and covered more advanced topics. I wanted to share all my knowledge with the students and I thought that would be great. However, after a few sessions, I noticed that the students couldn’t answer my questions, which I thought were easy. Therefore, in the next session, I asked them for their feedback on the course.
I received the following feedback from students:\\

“\emph {Thank you for going the extra mile to ensure that we understand the material in depth, but we are confused and we lose track of what is essential and what is additional information just to support the course content.}” \\

I realized that it was not very helpful to go beyond the necessary materials for the course. It only caused confusion and difficulty for the students to understand the main topics. I revised the course materials and removed the extra ones. After a few more sessions, I asked the students for their feedback. They were happy that the course only covered the basic materials, but they still had difficulty because I explained the topics in a complex way. I realized that I should have considered their age and level of class.

Frankly, my teaching performance was below my expectations the first time I taught this class. This is evident in the average score that I received for the class, which was a 15/20. The department’s average score for this course in the past years was a 18/20.The exams posed a challenge with questions that required a novel approach, designed by me. Subsequently, in the course review, I received feedback stating, \\

" \emph { The exam questions are not covered by the teacher.}"\\

Although the topics were indeed covered, some questions demanded a deeper understanding and a certain level of creativity in responses, which I realized might be too much for high school students.
Conducting a root cause analysis to identify the issues, I identified three main points:
\begin{itemize}
 \item   I deviated too much from the main course topics and covered unnecessary material.
  \item    I explained topics in a detailed and complex manner.
  \item    The exams were excessively difficult for high school students.
\end{itemize}
The key lesson learned is the need to focus more on the main course topics and use simpler language to explain the materials. Exam questions should also align more closely with the covered topics.\\
{\bf University Seasonal Lecturer:}\\ 

My next teaching experience was as a seasonal lecturer at a university. This was very different from teaching high school students, as the audience and the course were more advanced. Here, the students had chosen to take this course this semester, and they had the option to delay it until later in their program. The first course I taught was computer architecture. When I prepared the course materials, I applied the lesson I had learned from teaching in high school. I tried to focus on the essential concepts and use simple language. Since my audience was college students, I avoided repeating basic concepts and only covered the course materials. I hoped that this course would go better than the previous one. During the course, I noticed that the students asked questions, which I saw as a positive sign compared to my last class where I hardly got any questions. I generally felt that the students were engaged. After a few sessions, I asked the students for their feedback again. This time, the feedback was mainly about the gap in the topics that I assumed the students already knew.    Some of the feedbacks I received was such as:\\

“\emph {You seem to be very excited about sharing your knowledge with the students, but you assume that we already know some things. Some of these concepts were not taught before and this makes the lectures hard to follow.}”\\

“\emph {You don’t explain the topics in a simple and clear language.}”\\

“\emph {Using examples would make it easier to understand.}”\\

Based on this feedback, I updated the course materials and added more examples and simpler explanations. I also noted the concepts that I expected the students to know and checked their familiarity with them before proceeding with the course. The exam grades improved compared to my previous experience and the students achieved an average of 16/20, which was the usual grade for this course. At the end of the class, I received more feedback from the students. Some of the issues they raised were: \\

“\emph { speaking too quickly, reading directly from the notes and not making eye contact, giving unclear definitions and examples}.\\

 When I reviewed my class, I looked at the feedback again and thought about what I could do better. I realized that I had to understand who I was teaching. I had to know what year they were in, what they had learned before, and what courses they had taken. I had overestimated the students’ level. So, the first thing I did was to adjust the course materials and include the concepts that had not been taught before. The next thing I did was to prepare a student course packet that had a more detailed outline of my lectures, diagrams, charts, and overheads. This course packet was more comprehensive and covered the concepts that were necessary to understand the course topics.
 
 After implementing changes, I revised the information on slides, allowing for a higher-level explanation of the course. This approach clarified the concepts and the reasons behind the materials we were studying. Upon revisiting the course, I observed a notable reduction in student confusion, as they became more aware of what was crucial to understand and what details were less critical.

To enhance comprehension, I included simple examples into the slides, which proved highly effective in helping students' understanding of the concepts. Additionally, I introduced weekly assignments featuring more intricate problems. This strategy provided students with the opportunity to tackle complex questions and foster creativity in finding solutions.

To balance assessment, I allocated marks between exams and assignments. As a result, I received overwhelmingly positive feedback this time. Despite the exams containing more challenging questions, the inclusion of assignments allowed students to successfully tackle these complexities. Consequently, overall grades improved compared to the previous semester.

In my role as a teaching assistant, I created materials and tutorials, answering student questions during labs. I helped students develop ideas into flowcharts and then into codes for implementation on boards like FPGA, Arduino, or Raspberry Pi, offering support in debugging.
This experience deepened my understanding of  including diversity and inclusion concepts. I assisted students during labs, office hours, and outside my assigned times, responding to their questions through email or in-person interactions. This is an example that even while preparing for my final exam, I set aside time for students.
%\emph{
%    Hello Moslem,\\
%    We, group 16, were wondering if you could set aside some time tomorrow to view our code for lab {\#}3. Our code compiles properly and test bench gives adequate results, however we are struggling to get a correct output on the FPGA board. We would really appreciate some assistance.\\
%    Best regards,\\\\
%Hello,\\
%Sure. I have final exam myself but I can put aside some time.   What time work for you?\\
%Moslem Heidarpur,}


During this period, the professor, who was originally teaching the course, requested me to conduct a session in his absence. \\
{\itshape
Hi Moslem:


If you are available and willing, please consider giving a lecture on VHDL in my next Monday (Jan. 14)

from 1 to 2:20. I have an important meeting with industry people at 11:30 and I may not be able to start 

the class on time. If you can do this, please let me know ASAP.


It is totally up to you, you don't have to give this lecture if you are busy.}\\

I gladly accepted the professor's request and promptly started preparing for the session, based on my prior experiences.
Having served as a teaching assistant for the same course, I was already acquainted with the students. The session exceeded my expectations, allowing for a more  discussion of theory rather than focusing solely on lab work. Despite being a single session and without creating course materials, I successfully covered all the necessary course content. Overall, I received positive feedback from students during my time as a teaching assistant at the University of Windsor.


Subsequently, I held a postdoctoral position at the University of Windsor, collaborating with master's and PhD. students, offering assistance with their projects. I shared the knowledge I had gathered during my PhD, aiding them in debugging their code and reviewing their papers. The outcomes were highly positive, as these students successfully graduated with a commendable record of publications. Throughout this period, I acquired valuable experience in leadership and mentorship.


This marked my initial experience as an instructor in Canada. Despite not being responsible for creating course materials and labs, I tried to introduce some changes, incorporating a few new labs to align the course more closely with my intended coverage.

After two sessions, I sought feedback from students through two questions:

    What do you think I did well and would like me to continue doing?
    What do you think I can do better, and what suggestions do you have for improvement?

Regarding the positive feedback, students appreciated the way I explained the topics in a clear and understandable language.\\


\emph{" slides seems complicated but the way that you explain it make it much easier"}\\

One of the positive feedbacks that I received was about the live drawing that I did in the class. From my experience teaching in college, I knew that I should not take for granted that students know basic concepts that may seem very simple to me. So, I always asked students if they were familiar with the concepts or not and if they were not, I used my skills with Inkscape to make quick drawings and explain the concepts.


Students were also positive about Real-world examples that I did during the class. For instance, when I discussed the concept of “Whaling” as a type of social engineering attack, I gave a real-life example of how in 2019, an attacker pretended to be the CEO of an Austrian aerospace manufacturer and fooled the company into sending 42 million euros to their account. I showed the page that described the attack and shared the link in the meeting chat. I believe such examples really helped to interest the students and they were very useful in involving them in the course.





Positive feedback included appreciation for my ability to explain topics in an easy and understandable language, even when the slides were perceived as complex. I was pleased to have learned from past experiences that no concept is inherently too complex, and it ultimately depends on the instructor's experience and understanding to convey it in an accessible manner. The live drawing during class received praise as well, aiding in clarifying concepts and not assuming familiarity with basic ideas.

Real-world examples, such as illustrating the concept of "Whaling" with a specific case, were also positively received. Providing concrete examples, like the 2019 impersonation attack on an aerospace manufacturer, engaged students and enhanced their understanding of the course content.

I also received some suggestions to improve the course. The main issue was that the class was not interactive and I was talking most of the time. This was new for me as this was not a problem for the students in my previous classes and I quickly realized that they were right and this was something I completely overlooked. After the session, I thought about it and made a major change and later in the session I increased the student participation in the class significantly. When I wanted to introduce a new topic, I tried to help students understand why we were learning that topic and what was the challenge and how the technique taught in the topic could be a solution to the challenge. I asked them to share if they had any experience with any of the attacks or security measures and tools discussed in the course and to my surprise some of them had very good experience and by sharing their examples they enhanced the course greatly.Overall, after implementing these changes, the class received very positive written comments from students, indicating a successful adaptation to their feedback.\\


\emph{"Thanks so much Moslem. Enjoyed every minute. Fantastic job."}\\

\emph{"Thank you Moslem good stuff"}\\

\emph{"Thank you Moslem! Really enjoyed the session"}\\

\emph{"Thank you very much for all of your effort with us"}\\


In conclusion, my journey to become a better teacher has taken me from receiving disappointing feedback and doubting my teaching skills when I was a high school teacher to now where I have a fair amount of experience and my last teaching session was not flawless but it was close to what I envisioned. Thanks for your consideration 




