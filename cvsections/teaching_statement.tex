{\bf Teaching Statement}
\vspace*{3\baselineskip}\\
{\bf Inspiration}

My father was a teacher. He always inspired me to learn by asking questions and intriguing my sense of curiosity. The same sense of curiosity  that grew my motivation to continue  my education and become a researcher. One quality that I most like about him and later, became a part of my character as well, was his critical way of thinking and teaching.  Instead of giving direct answers to my questions, he guided me and encouraged me to discover the answers for myself.
 From what I learned, it is the disciplined practice of critical thinking that
enables  students to examine ideas, determine their validity and become creative.  
In short,  a great teacher provides students with the motivation toward becoming a knowledgeable and innovative person.
My approach to teaching
 is based on same idea of critical thinking and has three main components: Inquiry,  a deep understanding, and novelty.
 \\ [0.2cm]
{\bf Inquiry }

I believe that the best way  to encourage students is inquiry-based learning. 
Students tendency toward a new concept will be higher when they believe that it can be answer to a problem in their minds.  
The idea is to start the class with a few well-chosen questions and have the students explore the topic themselves, with guidance from the teacher. 
In this method, I will express the fundamental problem,
 encourage my students with a desire to know the answer and then  let the class  discusses it. 
 My role would be more an moderator who leads students in a challenging experience  planned to promote discussion, thinking and discovery. As an example,  when I started Digital Logic Design at  beginning of the semester and intended to talk about number systems,  I initiated the class with the questions  "why predominant number system is decimal and why not for instance, octal or hexadecimal?"
 And after discussing that, I asked the next question "now, why machines use binary number system?".  I observed that larger number of students engaged in answering the second question, when the first question was discussed in advance, compared to classes when I went directly for the second questions. 

%As an example, I when I started Digital Logic Design at  beginning of the semester and intended to talk about number systems,  I initiated class with the question  "why the predominant number system is decimal and why not for instance, octal or hexadecimal?"
%I observed that these question intrigues students sense of curiosity.
% And after that I asked another question "why machines use binary number system?".
% I observed that number of  students  participate in the class  answering this question increase when the first question asked before this. 

 
 
 This method also leads to conceptual understanding of the the problems and solutions in electrical engineering.    
   Students with conceptual understanding know more than solitary facts and methods and are able to  connect different concepts, decipher problems,  and apply engineering methods in novel ways. \\[0.2cm]
   {\bf A Deep Understanding}
   
   A good teaching method is a combination of different  techniques. Another technique that I find to be most useful is  cognitive learning where learning focus is on understating the 
subject at the deeper level rather than memorization of approaches toward certain problems.
  My approach toward this process is based on the following elements:  First, students need to understand why they are learning the subject. Second, is gaining knowledge about the subject at a deep level and third, is to contemplate on the application of what they learned. 
 The course contents need to be designed in a way that the students get a deep understanding  and develop the skills to apply their knowledge in their field. 
    
    An example is when I was presenting logical operations (AND, OR and NOT) at a Digital Logic Design class. First, I explained the concept of idealized switches network 
    where switches are considered as having only two exclusive states of open or close. Further, I gave them simple examples and  asked them    to identify different combinations and make separate components.      Later, I  introduced the Boolean algebra and logical
operations as a way to analyze and design circuits by algebraic means in terms of logic gates. \\[0.2cm]
   {\bf Creativity}
   
Perhaps the most important part of learning  and at the highest level, is creativity. 
    The scientific approach toward creativity involves studying  and mastering what have been achieved so far, asking  questions, proposing  solutions, designing experiments to check validity of the solutions and finally communicating the research with other scientists to have feedbacks. 
    
    My plan to engage graduate  and last year undergraduate students in research would be by means of the following. First, I will give them enough background and knowledge in a specific area which I believe have potential for research. Further, I will ask them to identify and make a list of the new and prominent works published in the subject. Afterwards, I will ask them to write a short brief on those papers and identify  their contributions. The next step would be presentation of the papers that foster deeper understanding of the works. Personally, I believe explaining something to someone else helps to have better insight about the topic.  Therefore, such conferences would be an fundamental part of my approach which  helps students both to evaluate the literature and communicating their findings.    
    
    The next move would be critical analysis of the papers. Students need to think about the weaknesses, limitations and drawbacks of the proposed approaches. Performing simulations in order to   replicating and testing findings on the papers also helps them to 
    learn the required tools and approaches toward the problems and also facilitate spotting the problems.
    
    At end of any phase, it is my duty as a teacher to evaluate their work and give them feedbacks and suggestions to improve their work. This will give students the opportunity to improve their ability to identify the problem, review the literature  for solutions, propose new solutions and finally communicating their research. 
    
    
    In summary, I wish to give back those qualities that meant the most to me during my
education: inquiry, a deep understanding and creativity. I hope to develop the love of engineering in my students since personally, I feel an infinite passion for it. As a saying states: "Choose a career you love and you will never have to work a day in your life".
